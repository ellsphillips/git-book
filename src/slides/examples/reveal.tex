\newcommand\examplebulletlistlength{3}
\newlength{\exampleleftListLength}
\setlength{\exampleleftListLength}{0.45\textwidth}
\newlength{\examplerightListLength}
\setlength{\examplerightListLength}{0.45\textwidth}

\begin{frame}
  \frametitle{Adaptive brace}
	\monocolumn{
		\begin{tikzpicture}[
      baseline=2.5ex,
      every node/.style={
        text depth=0.2ex,
        text height=1.3ex
      }
    ]
      
      \matrix[
        matrix of math nodes,
        column 1/.style={
          anchor=base west,
          nodes={font=\small}
        },
      ] (list1) {
        \text{Item 1} \\
        \text{\onslide<2->{Item 2}} \\
        \text{\onslide<3->{Item 3}} \\
        \text{\color{pink}\rule{\exampleleftListLength}{1mm}} \\
      };

	\foreach \pos in {1, 2, ..., \examplebulletlistlength} {%
		\pgfmathparse{1-\pos*\examplebulletlistlength/100+.05}
		\edef\bracepos{\pgfmathresult}
		\only<\pos>{%
			\draw[varbrace=\bracepos]%
				(list1-\the\numexpr\examplebulletlistlength+1\relax-1.north west)%
				-- %
				(list1-1-1.north west-|list1-\the\numexpr\examplebulletlistlength+1\relax-1.south west);
		}
	}
      

      \node[left=0\textwidth of list1, xshift = -.25cm, yshift = -1.75cm] (list2) {
      	\begin{tikzpicture}
      		\foreach \examplerightcontent [count=\exampleslideNumber from 1] in {
      			{item1},%
      			{item2},%
      			{item3},%
      		}{
	      		\only<\exampleslideNumber>{%
	      			\node (leftcontent) {
	      				\begin{varwidth}{\examplerightListLength}
	      					\begin{itemize}
	      						\small
	      						\only<1>{
	      							\item First lot of content
	      							\item First lot of content
	      						}
	      						\only<2>{
	      							\item Content on second slide
	      							\begin{itemize}
	      								\item Content on second slide
	      							\end{itemize}
      								\item Content on second slide
	      						}
	      						\only<3>{
	      							\item Final slide of content
	      							\item Final slide of content
	      							\item Final slide of content
	      							\item Final slide of content
	      						}
	      					\end{itemize}
	      					\vskip -.33\baselineskip
	      					\color{orange}\rule{\examplerightListLength}{1mm}
	      				\end{varwidth}
      				};
      			}
	      	}
      	\end{tikzpicture}
      };
    \end{tikzpicture} 
	}

  \note {
    %
  }
\end{frame}